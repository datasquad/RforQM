\documentclass[]{article}
\usepackage{lmodern}
\usepackage{amssymb,amsmath}
\usepackage{ifxetex,ifluatex}
\usepackage{fixltx2e} % provides \textsubscript
\ifnum 0\ifxetex 1\fi\ifluatex 1\fi=0 % if pdftex
  \usepackage[T1]{fontenc}
  \usepackage[utf8]{inputenc}
\else % if luatex or xelatex
  \ifxetex
    \usepackage{mathspec}
  \else
    \usepackage{fontspec}
  \fi
  \defaultfontfeatures{Ligatures=TeX,Scale=MatchLowercase}
\fi
% use upquote if available, for straight quotes in verbatim environments
\IfFileExists{upquote.sty}{\usepackage{upquote}}{}
% use microtype if available
\IfFileExists{microtype.sty}{%
\usepackage{microtype}
\UseMicrotypeSet[protrusion]{basicmath} % disable protrusion for tt fonts
}{}
\usepackage[margin=1in]{geometry}
\usepackage{hyperref}
\hypersetup{unicode=true,
            pdftitle={Introduction to Data Handling and Statistics 2},
            pdfauthor={Ralf Becker},
            pdfborder={0 0 0},
            breaklinks=true}
\urlstyle{same}  % don't use monospace font for urls
\usepackage{color}
\usepackage{fancyvrb}
\newcommand{\VerbBar}{|}
\newcommand{\VERB}{\Verb[commandchars=\\\{\}]}
\DefineVerbatimEnvironment{Highlighting}{Verbatim}{commandchars=\\\{\}}
% Add ',fontsize=\small' for more characters per line
\usepackage{framed}
\definecolor{shadecolor}{RGB}{248,248,248}
\newenvironment{Shaded}{\begin{snugshade}}{\end{snugshade}}
\newcommand{\KeywordTok}[1]{\textcolor[rgb]{0.13,0.29,0.53}{\textbf{#1}}}
\newcommand{\DataTypeTok}[1]{\textcolor[rgb]{0.13,0.29,0.53}{#1}}
\newcommand{\DecValTok}[1]{\textcolor[rgb]{0.00,0.00,0.81}{#1}}
\newcommand{\BaseNTok}[1]{\textcolor[rgb]{0.00,0.00,0.81}{#1}}
\newcommand{\FloatTok}[1]{\textcolor[rgb]{0.00,0.00,0.81}{#1}}
\newcommand{\ConstantTok}[1]{\textcolor[rgb]{0.00,0.00,0.00}{#1}}
\newcommand{\CharTok}[1]{\textcolor[rgb]{0.31,0.60,0.02}{#1}}
\newcommand{\SpecialCharTok}[1]{\textcolor[rgb]{0.00,0.00,0.00}{#1}}
\newcommand{\StringTok}[1]{\textcolor[rgb]{0.31,0.60,0.02}{#1}}
\newcommand{\VerbatimStringTok}[1]{\textcolor[rgb]{0.31,0.60,0.02}{#1}}
\newcommand{\SpecialStringTok}[1]{\textcolor[rgb]{0.31,0.60,0.02}{#1}}
\newcommand{\ImportTok}[1]{#1}
\newcommand{\CommentTok}[1]{\textcolor[rgb]{0.56,0.35,0.01}{\textit{#1}}}
\newcommand{\DocumentationTok}[1]{\textcolor[rgb]{0.56,0.35,0.01}{\textbf{\textit{#1}}}}
\newcommand{\AnnotationTok}[1]{\textcolor[rgb]{0.56,0.35,0.01}{\textbf{\textit{#1}}}}
\newcommand{\CommentVarTok}[1]{\textcolor[rgb]{0.56,0.35,0.01}{\textbf{\textit{#1}}}}
\newcommand{\OtherTok}[1]{\textcolor[rgb]{0.56,0.35,0.01}{#1}}
\newcommand{\FunctionTok}[1]{\textcolor[rgb]{0.00,0.00,0.00}{#1}}
\newcommand{\VariableTok}[1]{\textcolor[rgb]{0.00,0.00,0.00}{#1}}
\newcommand{\ControlFlowTok}[1]{\textcolor[rgb]{0.13,0.29,0.53}{\textbf{#1}}}
\newcommand{\OperatorTok}[1]{\textcolor[rgb]{0.81,0.36,0.00}{\textbf{#1}}}
\newcommand{\BuiltInTok}[1]{#1}
\newcommand{\ExtensionTok}[1]{#1}
\newcommand{\PreprocessorTok}[1]{\textcolor[rgb]{0.56,0.35,0.01}{\textit{#1}}}
\newcommand{\AttributeTok}[1]{\textcolor[rgb]{0.77,0.63,0.00}{#1}}
\newcommand{\RegionMarkerTok}[1]{#1}
\newcommand{\InformationTok}[1]{\textcolor[rgb]{0.56,0.35,0.01}{\textbf{\textit{#1}}}}
\newcommand{\WarningTok}[1]{\textcolor[rgb]{0.56,0.35,0.01}{\textbf{\textit{#1}}}}
\newcommand{\AlertTok}[1]{\textcolor[rgb]{0.94,0.16,0.16}{#1}}
\newcommand{\ErrorTok}[1]{\textcolor[rgb]{0.64,0.00,0.00}{\textbf{#1}}}
\newcommand{\NormalTok}[1]{#1}
\usepackage{graphicx,grffile}
\makeatletter
\def\maxwidth{\ifdim\Gin@nat@width>\linewidth\linewidth\else\Gin@nat@width\fi}
\def\maxheight{\ifdim\Gin@nat@height>\textheight\textheight\else\Gin@nat@height\fi}
\makeatother
% Scale images if necessary, so that they will not overflow the page
% margins by default, and it is still possible to overwrite the defaults
% using explicit options in \includegraphics[width, height, ...]{}
\setkeys{Gin}{width=\maxwidth,height=\maxheight,keepaspectratio}
\IfFileExists{parskip.sty}{%
\usepackage{parskip}
}{% else
\setlength{\parindent}{0pt}
\setlength{\parskip}{6pt plus 2pt minus 1pt}
}
\setlength{\emergencystretch}{3em}  % prevent overfull lines
\providecommand{\tightlist}{%
  \setlength{\itemsep}{0pt}\setlength{\parskip}{0pt}}
\setcounter{secnumdepth}{0}
% Redefines (sub)paragraphs to behave more like sections
\ifx\paragraph\undefined\else
\let\oldparagraph\paragraph
\renewcommand{\paragraph}[1]{\oldparagraph{#1}\mbox{}}
\fi
\ifx\subparagraph\undefined\else
\let\oldsubparagraph\subparagraph
\renewcommand{\subparagraph}[1]{\oldsubparagraph{#1}\mbox{}}
\fi

%%% Use protect on footnotes to avoid problems with footnotes in titles
\let\rmarkdownfootnote\footnote%
\def\footnote{\protect\rmarkdownfootnote}

%%% Change title format to be more compact
\usepackage{titling}

% Create subtitle command for use in maketitle
\newcommand{\subtitle}[1]{
  \posttitle{
    \begin{center}\large#1\end{center}
    }
}

\setlength{\droptitle}{-2em}

  \title{Introduction to Data Handling and Statistics 2}
    \pretitle{\vspace{\droptitle}\centering\huge}
  \posttitle{\par}
    \author{Ralf Becker}
    \preauthor{\centering\large\emph}
  \postauthor{\par}
      \predate{\centering\large\emph}
  \postdate{\par}
    \date{29 November 2018}


\begin{document}
\maketitle

\section{Preparing your workfile}\label{preparing-your-workfile}

We add the basic libraries needed for this week's work:

\begin{Shaded}
\begin{Highlighting}[]
\KeywordTok{library}\NormalTok{(tidyverse)    }\CommentTok{# for almost all data handling tasks}
\KeywordTok{library}\NormalTok{(readxl)       }\CommentTok{# to import Excel data}
\KeywordTok{library}\NormalTok{(ggplot2)      }\CommentTok{# to produce nice graphiscs}
\KeywordTok{library}\NormalTok{(stargazer)    }\CommentTok{# to produce nice results tables}
\end{Highlighting}
\end{Shaded}

\section{Introduction}\label{introduction}

The example we are using here is taken from the CORE - Doing Economics
resource. In particular we are using Project 8 which deals with
international data on well-being. The data represent several waves of
data from the European Value Study (EVS). A wave means that the same
surevey is repeated at regular intervals (waves).

\section{Aim of this lesson}\label{aim-of-this-lesson}

In this lesson we will revise some hypothesis testing and basic (simple)
regression analysis.

\section{Importing Data}\label{importing-data}

The data have been prepared as demonstrated in the Doing Economics
Project 8, up to and including Walk-Through 8.3. Please have a look at
this to understand the amount of data work required before an empirical
analysis can begin. The datafile is saved as an R data structure
(wb\_data.Rdata).

\begin{Shaded}
\begin{Highlighting}[]
\CommentTok{#wb_data <- readRDS("wellbeing_data.RDS")   # load RDS file}
\KeywordTok{load}\NormalTok{(}\StringTok{"WBdata.Rdata"}\NormalTok{)}
\KeywordTok{str}\NormalTok{(wb_data)  }\CommentTok{# prints some basic info on variables}
\end{Highlighting}
\end{Shaded}

\begin{verbatim}
## Classes 'tbl_df', 'tbl' and 'data.frame':    129515 obs. of  19 variables:
##  $ S002EVS    : chr  "1981-1984" "1981-1984" "1981-1984" "1981-1984" ...
##  $ S003       : chr  "Belgium" "Belgium" "Belgium" "Belgium" ...
##  $ S006       : num  1001 1002 1003 1004 1005 ...
##  $ A009       : num  3 5 2 5 5 5 5 5 4 4 ...
##  $ A170       : num  9 9 3 9 9 9 9 10 8 10 ...
##  $ C036       : num  NA NA NA NA NA NA NA NA NA NA ...
##  $ C037       : num  NA NA NA NA NA NA NA NA NA NA ...
##  $ C038       : num  NA NA NA NA NA NA NA NA NA NA ...
##  $ C039       : num  NA NA NA NA NA NA NA NA NA NA ...
##  $ C041       : num  NA NA NA NA NA NA NA NA NA NA ...
##  $ X001       : chr  "Male" "Male" "Male" "Female" ...
##  $ X003       : num  53 30 61 60 60 19 38 39 44 76 ...
##  $ X007       : chr  "Single/Never married" "Married" "Separated" "Married" ...
##  $ X011_01    : num  NA NA NA NA NA NA NA NA NA NA ...
##  $ X025A      : chr  NA NA NA NA ...
##  $ Education_1: num  NA NA NA NA NA NA NA NA NA NA ...
##  $ Education_2: chr  NA NA NA NA ...
##  $ X028       : chr  "Full time" "Full time" "Unemployed" "Housewife" ...
##  $ X047D      : num  NA NA NA NA NA NA NA NA NA NA ...
\end{verbatim}

Checking your environment you will see two objects. Along the proper
datafile (\texttt{wb\_data}) you will find \texttt{wb\_data\_Des} which
contains some information for each of the variables. It will help us to
navigate the obscure variable names.

\begin{Shaded}
\begin{Highlighting}[]
\NormalTok{wb_data_Des}
\end{Highlighting}
\end{Shaded}

\begin{verbatim}
##          Names                   Labels
## 1      S002EVS                 EVS-wave
## 2         S003           Country/region
## 3         S006        Respondent number
## 4         A009                   Health
## 5         A170        Life satisfaction
## 6         C036                  Work Q1
## 7         C037                  Work Q2
## 8         C038                  Work Q3
## 9         C039                  Work Q4
## 10        C041                  Work Q5
## 11        X001                      Sex
## 12        X003                      Age
## 13        X007           Marital status
## 14     X011_01       Number of children
## 15       X025A                Education
## 16 Education_1       Education category
## 17 Education_2    Education Description
## 18        X028               Employment
## 19       X047D Monthly household income
##                                                                                              Description
## 1                                                                                               EVS-wave
## 2                                                                                         Country/region
## 3                                                                             Original respondent number
## 4                                             State of health (subjective), 1 = Very Poor, 5 = Very good
## 5                                                                            Satisfaction with your life
## 6                   To develop talents you need to have a job, 1 = Strongly Agree, 5 = Strongly Disagree
## 7  Humiliating to receive money without having to work for it, 1 = Strongly Agree, 5 = Strongly Disagree
## 8                           People who don't work become lazy, 1 = Strongly Agree, 5 = Strongly Disagree
## 9                              Work is a duty towards society, 1 = Strongly Agree, 5 = Strongly Disagree
## 10    Work should come first even if it means less spare time, 1 = Strongly Agree, 5 = Strongly Disagree
## 11                                                                                                   Sex
## 12                                                                                                   Age
## 13                                                                                        Marital status
## 14                                             How many children you have-deceased children not included
## 15                                                    Educational level respondent: ISCED-code one digit
## 16                                                                      Educational ISCED-code one digit
## 17                                                                           Education level description
## 18                                                                                     Employment status
## 19                                        Monthly household income (? 1,000), corrected for ppp in euros
\end{verbatim}

As you can see there are a number of interesting questions in this
dataset. These questions will allow us to investigate whether attitudes
to work differ between coountries and whether such differences correlate
to different levels of self-reported happiness.

\section{Some initial data analysis and summary
statistics}\label{some-initial-data-analysis-and-summary-statistics}

Let us investigate some of the features of this dataset. It has 129515
observations and 19 variables. Let's see which countries are represented
in our dataset.

\begin{Shaded}
\begin{Highlighting}[]
\KeywordTok{unique}\NormalTok{(wb_data}\OperatorTok{$}\NormalTok{S003)   }\CommentTok{# unque finds all the different values in a variable}
\end{Highlighting}
\end{Shaded}

\begin{verbatim}
##  [1] "Belgium"            "Canada"             "Denmark"           
##  [4] "France"             "Germany"            "Iceland"           
##  [7] "Ireland"            "Italy"              "Malta"             
## [10] "Netherlands"        "Norway"             "Spain"             
## [13] "Sweden"             "Great Britain"      "United States"     
## [16] "Northern Ireland"   "Austria"            "Bulgaria"          
## [19] "Czech Republic"     "Estonia"            "Finland"           
## [22] "Hungary"            "Latvia"             "Lithuania"         
## [25] "Poland"             "Portugal"           "Romania"           
## [28] "Slovakia"           "Slovenia"           "Croatia"           
## [31] "Greece"             "Russian Federation" "Turkey"            
## [34] "Albania"            "Armenia"            "Bosnia Herzegovina"
## [37] "Belarus"            "Cyprus"             "Northern Cyprus"   
## [40] "Georgia"            "Luxembourg"         "Moldova"           
## [43] "Montenegro"         "Serbia"             "Switzerland"       
## [46] "Ukraine"            "Macedonia"          "Kosovo"
\end{verbatim}

As you can see these are 48 countries, almost all European, Canada and
the U.S. being the exceptions. In the same manner we can find out how
many waves of data we have available.

\begin{Shaded}
\begin{Highlighting}[]
\KeywordTok{unique}\NormalTok{(wb_data}\OperatorTok{$}\NormalTok{S002EVS)}
\end{Highlighting}
\end{Shaded}

\begin{verbatim}
## [1] "1981-1984" "1990-1993" "1999-2001" "2008-2010"
\end{verbatim}

Let's find out how many observations/respondents we have for each
country in each year. To do this we will resort to the powerful piping
technique delievered through the functionality of the \texttt{tidyverse}

\begin{Shaded}
\begin{Highlighting}[]
\NormalTok{table1 <-}\StringTok{ }\NormalTok{wb_data }\OperatorTok\StringTok{ }\KeywordTok{group_by}\NormalTok{(S002EVS,S003) }\OperatorTok\StringTok{ }
\StringTok{            }\KeywordTok{summarise}\NormalTok{(}\DataTypeTok{n =} \KeywordTok{n}\NormalTok{()) }\OperatorTok\StringTok{ }
\StringTok{            }\KeywordTok{spread}\NormalTok{(S002EVS,n) }\OperatorTok\StringTok{ }
\StringTok{            }\KeywordTok{print}\NormalTok{()}
\end{Highlighting}
\end{Shaded}

\begin{verbatim}
## # A tibble: 48 x 5
##    S003               `1981-1984` `1990-1993` `1999-2001` `2008-2010`
##    <chr>                    <int>       <int>       <int>       <int>
##  1 Albania                     NA          NA          NA        1200
##  2 Armenia                     NA          NA          NA        1224
##  3 Austria                     NA        1432          NA        1216
##  4 Belarus                     NA          NA          NA        1237
##  5 Belgium                   1025        2721        1402        1343
##  6 Bosnia Herzegovina          NA          NA          NA        1104
##  7 Bulgaria                    NA         984         858        1183
##  8 Canada                    1241        1717          NA          NA
##  9 Croatia                     NA          NA         849        1188
## 10 Cyprus                      NA          NA          NA         775
## # ... with 38 more rows
\end{verbatim}


\end{document}
